After compiling \dftfe{} as described above, we have now the \verb|real/main| executable, which uses real data-structures for the Kohn-Sham DFT eigen solve. This is sufficient for fully non-periodic problems, and periodic and semi-periodic problems with only one Brillouin zone sampling point at the origin. The other executable is \verb|complex/main|, which uses complex data-structrues for the Kohn-Sham DFT eigen solve. This is required for periodic and semi-periodic problems with multiple Brillouin zone sampling points. These executables are to be used as follows:
\begin{verbatim}
  ./main parameterFile.prm
\end{verbatim}
or, for a parallel program:
\begin{verbatim}
  mpirun -n N ./main parameterFile.prm
\end{verbatim}
to run with N processors. 
\subsection{Structuring the input file}
In the above, an input file with \verb|.prm| extension is used. This file contains input parameters as described in Section~\ref{sec:parameters}, which can be of multiple types (\verb|string, double, integer, bool etc.|). All input parameters are also conveniently indexed at the end of this manual in Section~\ref{sec:runtime-parameter-index-full}. There are two types of parameters: ``{\it Global parameters}" and ``{\it Parameters in section A/B/..}". This can be seen directly in Section~\ref{sec:parameters}, where each parameter belongs to a group of parameters under the headings: ``{\it Global parameters}" or ``{\it Parameters in section A/B/..}". In ``{\it Parameters in section A/B/..}", {\it A} refers to the primary subsection name, {\it B} if present refers to a subsection inside {\it A}, and so on. 

First, lets consider how to use a parameter named \verb|PARAMETER xyz| under the heading ``{\it Global parameters}". To set it to a value, say \verb|value|  in the  \verb|.prm| file, directly use
\begin{verbatim}
  set PARAMETER xyz=value
\end{verbatim}
Next consider a parameter named \verb|PARAMETER xyzA| under the heading ``{\it Parameters in section A}". To set it to a value, say \verb|value|  in the  \verb|.prm| file, use 
\begin{verbatim}
subsection A
  set PARAMETER xyzA=value
end
\end{verbatim}
Finally, consider a nested parameter named  \verb|PARAMETER xyzAB| under the heading ``{\it Parameters in section A/B}". To set it to a value, say \verb|value|  in the  \verb|.prm| file, use 
\begin{verbatim}
subsection A
  subsection B
    set PARAMETER xyzAB=value
  end
end
\end{verbatim}
Couple of final notes- more than one parameter could be used inside the same \verb|subsection|. For example
\begin{verbatim}
subsection A
  set PARAMETER SUBSECTION xyzA1=value1
  set PARAMETER SUBSECTION xyzA2=value2
  subsection B
    set PARAMETER SUBSUBSECTION xyzAB1=value1
    set PARAMETER SUBSUBSECTION xyzAB2=value2
  end
end
\end{verbatim}
, and indentation used in the above examples is only for readability.
\subsection{Demo examples walkthrough}
\subsubsection{Example 1}
Please follow the examples in the \verb|/dftfe/demo/| folder.

\subsubsection{Example 2}
To be written
